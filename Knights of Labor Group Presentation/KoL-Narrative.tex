\documentclass[a4paper, 12pt]{article}
\usepackage[a4paper, margin=1in]{geometry}
\usepackage[T1]{fontenc}
\usepackage[utf8]{inputenc}
\usepackage{lmodern}
\usepackage[american]{babel}
\usepackage{csquotes}
\usepackage{setspace}
\usepackage[notes,backend=biber]{biblatex-chicago}
\bibliography{references}
\newcommand{\HRule}{\rule{\linewidth}{0.25mm}}
\doublespacing
\setlength\parindent{36pt}
\begin{document}
\begin{titlepage}
\begin{center}

% Upper part of the page. The '~' is needed because \\
% only works if a paragraph has started.


%\textsc{\LARGE Arizona State University}\\[1.5cm]

~\\[2.5cm]

% Title
\HRule \\[0.4cm]
{ \Large \bfseries Schenck v U.S.}\\
{   A Supreme Court Majority Opinion}\\[0.4cm]

\HRule \\[0.5cm]

% Bottom of the page

%\vfill

\textbf{Abstract} \\

\begin{flushleft}
\begin{spacing}{1.0}

{\small
In the case of Schenck v. United States (1919), the Supreme Court was tasked with determining the constitutionality of the Espionage and Sedition Acts, which were enacted during World War I to protect the war effort. This document aims to weigh the evidence and determine the innocence of Charles Schenck as expressed by a majority ruling of the Court.}\\

\end{spacing}
\end{flushleft}

\vfill

\begin{flushright}
\small {Goyal, Anish \\}
\large AP United States History \\
\large Anslie Spitler\\
{\large March 13, 2023}
\end{flushright}

\end{center}
\end{titlepage} 
\section{Origin}
\hspace{\parindent} The Knights of Labor (KoL) union was founded on Thanksgiving in 1869 in Philadelphia by Uriah S. Stephens.\autocite{history.com} The organization's principal aim was to unite all workers, regardless of skill, in pursuit of higher wages and better working conditions. Stephens, a former indentured labor, believed that workers at a single factory were not enough to make changes and a more expansive organization was needed.\autocite{Foner1947} He welcomed all white male laborers into his new organization but excluded bankers, lawyers, gamblers, and saloon keepers because they were not viewed as "producing class workers."\autocite{Green1980} The KoL was also anti-Chinese,\autocite{history17b} as they viewed Chinese immigrants as "competition" that employers would use to drive down wages. After Stephens stepped down in 1879, Terrence Powderly, a machinist of Catholic Irish ancestry from Carbondale, Pennsylvania, took over the KoL.
\section{Accomplishments}
\hspace{\parindent} The knights of labor had many feats that they accomplished during their time as a union. Under Powderly's leadership, the Knights of Labor allowed women to join the organization as members with equal rights as men, helping it gain 700,000 members. One of the most notable accomplishments of the KoL was the Great Southwest Railroad Strike of 1886, which spanned four states: Missouri, Kansas, Arkansas, and Texas. The KoL, led by Powderly, battled the Gould System,\autocite{afl} headed by Jay Gould, for better wages and treatment for workers. The strike began on March 1, 1886 when a member of the KoL was fired from the Gould System for initiating a union meeting. Strikers of the KoL then boarded a train and intentionally killed the engine, and on March 20, they began burning railroad bridges in Texas.\autocite{RAILROAD1886}
\section{Decline}
\hspace{\parindent} However, internal disputes ultimately led to the decline of the Knights of Labor. The Haymarket Square Riot of 1886, an altercation between labor protestors and police that resulted in eleven deaths,\autocite{britannica} caused the KoL to be blamed for the incident.Additionally, the violent clashes between the people and police during the Great Southwest Railroad Strike, which was forced to end in May, resulted in a shift in public opinion against the KoL. Ultimately, the KoL's influence declined in the 1890s, as many of its members left to join other unions such as the American Federation of Labor.\autocite{Montgomery1987}
\section{Legacy}
\hspace{\parindent} Despite its decline, the Knights of Labor contributed to a greater awareness of issues such as child labor, equal pay for women, laws requiring employers to participate in arbitration to resolve differences with workers, and the nationalization of railroads and telephone networks.\autocite{Warren1951} Though the KoL did not achieve significant changes in laws, it was progressive by including immigrants and women, who were initially viewed as unskilled laborers, in the labor movement. The KoL also advocated for shorter working hours, standardizing the 8-hour workday that we see today and was later adopted by the American Federation of Labor. The KoL was one of the first American labor unions to admit both skilled and unskilled workers, as well as women and African Americans, making it one of the most inclusive and diverse unions of its time.\autocite{Lichtenstein1982}
\pagebreak
\printbibliography
\end{document}  