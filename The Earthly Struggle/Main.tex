\documentclass[stu]{apa7}
\usepackage{hyperref}
\usepackage{xcolor, soul}
\title{The Earthly Struggle}
\author{Anish Goyal}
\let\comma,
\authorsaffiliations{Gwinnett School of Math\comma{} Science\comma{} and Technology}
\course{American Literature and Composition}
\professor{Susan Kohanek}
\duedate{February 24, 2023}
\newcommand{\HRule}{\rule{\linewidth}{0.25mm}}
\setlength\parindent{36pt}
\authornote{The assignment was to use five texts of your choice to synthesize a found poem that effectively characterizes the Harlem Renaissance. There are two versions of the poem in this document: one that is highlighted according to the work from which each line was taken, and one that is not.}
\begin{document}
\maketitle
\tableofcontents
\newpage
\noindent\rule{\textwidth}{1pt}

\section{Highlighted Poem}
\sethlcolor{yellow}\hl{Yellow = ``America`` by Claude McKay}\\
\sethlcolor{green}\hl{Green = ``Harlem`` by Langston Hughes}\\
\sethlcolor{orange}\hl{Orange = ``A Poet to His Baby Son`` by James Weldon Johnson}\\
\sethlcolor{blue}\hl{Blue = ``Art vs. Trade`` by James Weldon Jonson}\\
\sethlcolor{red}\hl{Red = ``Georgia Dusk`` by Jean Toomer}\\
\vspace{5mm}

\hl{The sky, lazily disdaining to pursue,} \\
\hl{passively darkens for night`s barbecue---} \\
\hl{the} \underline{\hl{feast of moon}}\hl{and men and barking hounds.} \\
\hl{The sawmill blows its whistle,} \\
\hl{buzz-saws stop,} \\
\underline{\hl{and silence breaks the bud of knoll and hill.}}\footnote{Personifies silence by making it capable of breaking things, creating a vivid image.} \\
\vspace{5mm}
\sethlcolor{orange}
\hl{Tiny bit of humanity, cursed with your father`s mind;} \\
\hl{can it be that already you are thinking of being a poet?} \\
\hl{For poets no longer are makers of songs,} \\
\hl{sayers of the glories of earth and sky.} \\
\vspace{5mm}
\sethlcolor{blue}
\hl{Trade---trade versus art;} \\
\hl{brain---brain versus heart.} \\
\underline{\hl{Trade has spread out his mighty myriad claw,}}\footnote{A metaphor that compares trade to a creature with a large claw, emphasizing its destructive and dominating nature.} \\
\hl{and drawn into his foul polluted maw.} \\
\vspace{5mm}
\hl{O, the earthiness of these hard-hearted times,} \\
\hl{when} \underline{\hl{clinking dollars and jingling dimes}}\footnote[3]{Symbolism that represents larger concepts beyond their literal meanings. The feast of moon symbolizes nighttime and darkness while dollars and dimes represent materialism drowning out the finer, more creative aspects of life.} \\
\hl{drown all the finer music of the soul.} \\
\vspace{5mm}
\sethlcolor{yellow}
\hl{Yet---as a rebel fronts a king in state---} \\
\hl{I stand within her walls with not a shred} \\
\hl{of terror, malice,}  \\
\hl{not a word of jeer.} \\
\sethlcolor{orange}
\hl{My son, this is no time} \\
\hl{nor place for a poet.} \\
\vspace{5mm}
\sethlcolor{yellow}
\hl{Beneath the touch of Time`s unerring hand---} \\
{like priceless treasures sinking in the sand---} \\
\sethlcolor{blue}
\hl{poor art with struggling gasp, lies strangled,} \\
\hl{dying in his mighty grasp.} \\
\hl{O, if mankind had less of Brain and more of Heart!} \\
\vspace{5mm}
\sethlcolor{green}
\hl{What happens to a dream deferred?} \\
\hl{Does it dry up} \\
\hl{or fester like a sore?} \\
\hl{Does it stink like rotten meat} \\
\hl{or crust and sugar over?} \\
Let it be rescued, protected, defended; \\
\sethlcolor{yellow}
\hl{for the Earth is a cultured hell that tests our youth.} \\
\vspace{5mm}

\subsection{Explanatory Sentence}
This found poem was created using lines from five different poems written by prominent writers of the Harlem Renaissance: Claude McKay`s ``America,`` Langston Hughes` ``Harlem,`` James Weldon Johnson`s ``Art vs. Trade`` and ``A Poet to His Baby Son,`` and Jean Toomer`s ``Georgia Dusk.``

\subsection{Rationale Paragraph}
\indent The lines chosen for this found poem all touch on the theme of the struggle between art and commerce, or between creative expression and the demands of society. The lines from Claude McKay`s ``America`` and Jean Toomer`s ``Georgia Dusk`` depict the natural world as a source of inspiration and beauty that is threatened by the encroachment of modernity and industry. Langston Hughes` ``Harlem`` and James Weldon Johnson`s ``Art vs. Trade`` both express the tension between the desire to create and the need to make a living, with Johnson`s poem being more explicit in its condemnation of the dehumanizing effects of capitalism. Together, these lines create a powerful statement about the struggle of artists to maintain their integrity and creativity in a society that often seeks to exploit and commodify their talents. The choices made in selecting and arranging the lines were guided by a desire to capture the essence of this struggle and to create a cohesive and meaningful poem out of disparate sources. The result is a work that speaks to the ongoing importance of art in our lives and the need to protect and support it, even in the face of adversity and divisions of class, gender, race, etc. 

\newpage
\section{The Poem Without Highlights}

The sky, lazily disdaining to pursue, \\
passively darkens for night`s barbecue— \\
the feast of moon and men and barking hounds. \\
The sawmill blows its whistle, \\
buzz-saws stop, \\
and silence breaks the bud of knoll and hill. \\
\vspace{5mm}
Tiny bit of humanity, cursed with your father`s mind; \\
can it be that already you are thinking of being a poet? \\
For poets no longer are makers of songs, \\
sayers of the glories of earth and sky. \\
\vspace{5mm}
Trade—trade versus art;\\
brain—brain versus heart. \\
Trade has spread out his mighty myriad claw, \\
and drawn into his foul polluted maw. \\
\vspace{5mm}
O, the earthiness of these hard-hearted times, \\
when clinking dollars, and jingling dimes, \\
drown all the finer music of the soul. \\
\vspace{5mm}
Yet—as a rebel fronts a king in state— \\
I stand within her walls with not a shred \\
of terror, malice,  \\
not a word of jeer. \\
My son, this is no time \\
nor place for a poet. \\
\vspace{5mm}
Beneath the touch of Time`s unerring hand— \\
like priceless treasures sinking in the sand— \\
poor art with struggling gasp, lies strangled, \\
dying in his mighty grasp. \\
O, if mankind had less of Brain and more of Heart! \\
\vspace{5mm}
What happens to a dream deferred? \\
Does it dry up
or fester like a sore?
Does it stink like rotten meat
or crust and sugar over?
Let it be rescued, protected, defended; \\
for the Earth is a cultured hell that tests our youth. \\
\vspace{5mm}

\end{document}