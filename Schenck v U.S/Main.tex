\documentclass[a4paper, 12pt]{article}
\usepackage[a4paper, margin=1in]{geometry}
\usepackage[T1]{fontenc}
\usepackage[utf8]{inputenc}
\usepackage{lmodern}
\usepackage[american]{babel}
\usepackage{csquotes}
\usepackage{setspace}
\usepackage[notes,backend=biber]{biblatex-chicago}
\newcommand{\HRule}{\rule{\linewidth}{0.25mm}}
\doublespacing
\setlength\parindent{36pt}
\begin{document}
\begin{titlepage}
\begin{center}

% Upper part of the page. The '~' is needed because \\
% only works if a paragraph has started.


%\textsc{\LARGE Arizona State University}\\[1.5cm]

~\\[2.5cm]

% Title
\HRule \\[0.4cm]
{ \Large \bfseries Schenck v U.S.}\\
{   A Supreme Court Majority Opinion}\\[0.4cm]

\HRule \\[0.5cm]

% Bottom of the page

%\vfill

\textbf{Abstract} \\

\begin{flushleft}
\begin{spacing}{1.0}

{\small
In the case of Schenck v. United States (1919), the Supreme Court was tasked with determining the constitutionality of the Espionage and Sedition Acts, which were enacted during World War I to protect the war effort. This document aims to weigh the evidence and determine the innocence of Charles Schenck as expressed by a majority ruling of the Court.}\\

\end{spacing}
\end{flushleft}

\vfill

\begin{flushright}
\small {Goyal, Anish \\}
\large AP United States History \\
\large Anslie Spitler\\
{\large March 13, 2023}
\end{flushright}

\end{center}
\end{titlepage} 
\tableofcontents
\pagebreak
\section{Weighing the evidence}
\hspace{\parindent} Charles Schenck sent 15,000 leaflets to drafted men urging them to resist the draft, claiming that it violated the idea embodied in the 13th amendment to the Constitution of the United States and asserting that ``it was despotism and a monstrous wrong against humanity.`` He was then arrested for allegedly violating the Espionage and Sedition Acts of 1917 and 1918, which made it a crime to obstruct the recruitment or enlistment of persons into the military service of the United States, cause insubordination, disloyalty, mutiny, or refusal of duty in the military, and make false reports with intent to interfere with the operation or success of the military or naval forces of the United States. Despite this, Schenck believes that the leaflets should be protected as free speech and that the Espionage and Sedition Acts violate the First Amendment of the United States Constitution. 

\newpage
\section{Majority opinion}
\hspace{\parindent} The Supreme Court of the United States of America rules that Charles Schenck is indeed guilty of violating the Espionage and Sedition Acts. The evidence shows that Schenck was responsible for printing and distributing leaflets that urged resistance to the draft and obstructed the recruitment or enlistment service of the United States. The leaflets also contained false information, claiming that conscription was little better than slavery and that it benefited only the rich while causing suffering and death for the poor and working-class soldiers who would do the actual fighting in Europe.

\hspace{\parindent} Regarding the constitutionality of the Espionage and Sedition Acts, the Court has decided that the Acts did \emph{not} violate citizens' First Amendment rights. While the First Amendment protects freedom of speech, it does not protect speech that presents a clear and present danger to the nation's security or interferes with the war effort. In this case, Schenck's speech directly interfered with the recruitment and enlistment service of the United States during a time of war. Furthermore, the Espionage and Sedition Acts were necessary to protect the war effort and the safety of the nation. The laws prohibited false reports and statements that would interfere with the military or naval forces of the United States or promote the success of its enemies. They also prohibited inciting insubordination, disloyalty, mutiny, or refusal of duty, in the military or naval forces of the United States, as well as any disloyal, profane, scurrilous, or abusive language about the form of government of the United States, the Constitution of the United States, or the military or naval forces of the United States. The Court believes that these laws were necessary to protect the nation's security during a time of war, and breaking these laws would have infringed on the safety and security of other Americans.

\hspace{\parindent} The Court does hold, however, that the Espionage and Sedition Acts must be altered. While the Court understands that the federal government has a legitimate interest in protecting the country during wartime, the Acts do not make a clear distinction between speech that is protected and speech that is not, leaving too much room for interpretation and potentially chilling citizens` free speech. So, while the Court rules that the Espionage and Sedition Acts are constitutional, it also rules that they must be altered to make a clear distinction between protected and unprotected speech such that similar uprisings do not occur in the future.

\hspace{\parindent} In conclusion, the Supreme Court of the United States of America maintains that Charles Schenck is guilty of violating the Espionage and Sedition Acts, and the Espionage and Sedition Acts are not a violation of US citizens` First Amendment rights. The First Amendment is only valid when the right to free speech does not infringe on another civilian's fundamental liberties, such as the right to safety and security. For this reason, the Court believes that the Espionage and Sedition Acts were necessary to protect the nation`s security during a time of war and further alleges that they protected the fundamental rights of citizens in the United States. The Court does believe, however, that the Acts must be altered to make a clear distinction between protected and unprotected speech so that similar uprisings do not occur in the future.

\end{document}