\documentclass[a4paper, 12pt]{article}
\usepackage[a4paper, margin=1in]{geometry}
\usepackage[T1]{fontenc}
\usepackage[utf8]{inputenc}
\usepackage{lmodern}
\usepackage[american]{babel}
\usepackage{csquotes}
\usepackage{setspace}
\usepackage[notes,backend=biber]{biblatex-chicago}
\newcommand{\HRule}{\rule{\linewidth}{0.25mm}}
\doublespacing
\setlength\parindent{36pt}
\begin{document}
\begin{titlepage}
\begin{center}

% Upper part of the page. The '~' is needed because \\
% only works if a paragraph has started.


%\textsc{\LARGE Arizona State University}\\[1.5cm]

~\\[2.5cm]

% Title
\HRule \\[0.4cm]
{ \Large \bfseries The American Dairy Association:}\\
{   A Political Action Plan}\\[0.4cm]

\HRule \\[0.5cm]



% Bottom of the page

%\vfill

\textbf{Abstract} \\

\begin{flushleft}
\begin{spacing}{1.0}

{\small
The American Dairy Association proposes introducing a new variety of flavors of milk in schools across the country to increase students' milk intake and improve their overall health and nutrition while requiring the cooperation of all branches of government. The Association plans to work with the Executive and Legislative Branches to advocate for regulatory changes and legislation that supports their cause. The proposal also aims to create new markets for dairy farmers, boost the agricultural industry, and support local economies.}\\

\end{spacing}
\end{flushleft}

\vfill

\begin{flushright}
\small {Goyal, Anish \\
Kim, Brian \\
Nguyen, Luc \\}
\large AP United States History \\
\large Anslie Spitler\\
{\large February 15, 2023}
\end{flushright}


\end{center}
\end{titlepage} 
\tableofcontents
\pagebreak
\section{Who, What, Where, When, and Why?}
\hspace{\parindent} \textit{The 9/11 Commission Report} is a primary source that provides a comprehensive account of the events leading up to the September 11, 2001 terrorist attacks, the attacks themselves, and their aftermath. Published in 2004 by the National Commission on Terrorist Attacks Upon the United States, the report was the result of an extensive investigation that included interviews with more than 1,200 people and a review of millions of pages of documents. The report is a valuable primary source for future generations seeking to understand the impact of the 9/11 attacks on American society, politics, and foreign policy. It provides a detailed narrative of the events leading up to the attacks, including the intelligence failures and missed opportunities to prevent them. The report also outlines the government's response to the attacks, including the creation of the Department of Homeland Security and the War on Terror. \textit{The 9/11 Commission Report} is significant not only for its historical value but also for its impact on contemporary American politics and policy. The report's recommendations led to significant changes in national security policy, including the creation of the Director of National Intelligence and the National Counterterrorism Center. 
\end{document}