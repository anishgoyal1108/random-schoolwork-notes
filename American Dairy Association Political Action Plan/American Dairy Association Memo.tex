\documentclass[a4paper, 12pt]{article}
\usepackage[a4paper, margin=1in]{geometry}
\usepackage[T1]{fontenc}
\usepackage[utf8]{inputenc}
\usepackage{lmodern}
\usepackage[american]{babel}
\usepackage{csquotes}
\usepackage{setspace}
\usepackage[notes,backend=biber]{biblatex-chicago}
\newcommand{\HRule}{\rule{\linewidth}{0.25mm}}
\doublespacing
\setlength\parindent{36pt}
\begin{document}
\begin{titlepage}
\begin{center}

% Upper part of the page. The '~' is needed because \\
% only works if a paragraph has started.


%\textsc{\LARGE Arizona State University}\\[1.5cm]

~\\[2.5cm]

% Title
\HRule \\[0.4cm]
{ \Large \bfseries Schenck v U.S.}\\
{   A Supreme Court Majority Opinion}\\[0.4cm]

\HRule \\[0.5cm]

% Bottom of the page

%\vfill

\textbf{Abstract} \\

\begin{flushleft}
\begin{spacing}{1.0}

{\small
In the case of Schenck v. United States (1919), the Supreme Court was tasked with determining the constitutionality of the Espionage and Sedition Acts, which were enacted during World War I to protect the war effort. This document aims to weigh the evidence and determine the innocence of Charles Schenck as expressed by a majority ruling of the Court.}\\

\end{spacing}
\end{flushleft}

\vfill

\begin{flushright}
\small {Goyal, Anish \\}
\large AP United States History \\
\large Anslie Spitler\\
{\large March 13, 2023}
\end{flushright}

\end{center}
\end{titlepage} 
\tableofcontents
\pagebreak
\section{One Paragraph Summary}
\hspace{\parindent} The American Dairy Association would like to introduce a new variety of flavors of milk to schools across the country, benefitting both the farmers and the schools. Many students don`t drink milk solely because the taste doesn`t appeal to them, thereby missing out on the health benefits of milk. Milk has many crucial vitamins and nutrients such as calcium, protein, and vitamin D; students who refuse to drink milk often do not have a sufficient amount of these nutrients. Introducing different flavors could appeal to students, allowing them to both enjoy drinking milk and have a proper intake of nutrients.

\section{Actions Taken Through Government Branches} 
\subsection{The Executive Branch: The Presidency}
\hspace{\parindent} The presidency has significant power in policy-making, and there are several specific actions they can take to be effective, such as proposing legislation to Congress, issuing executive orders to direct the federal government to take specific actions without Congress, appointing officials who are aligned with policy priorities, and influencing policy outcomes by setting budgeting priorities. The most effective way for interest groups to influence the Executive Branch is to develop relationships with influential administration officials who can advocate for the interest group's policies and proposals within the Presidential administration. They can also use media outreach and public relations to draw attention to a specific cause, which causes the President to take action. Our goal is to meet with the President and relevant administration officials to discuss the benefits of introducing flavored milk and promote the American Dairy Association's interests as a whole. From there, we would develop relationships with influential administration members who can advocate for introducing flavored milk in schools and advance the association`s policy agenda. 

\subsection{The Executive Branch: The Bureaucracy}
\hspace{\parindent} The bureaucracy \emph{also} plays a huge role in policy-making, and they do this by building relationships with industry-standard experts to develop the best strategies for promoting policy agenda, advocating for changes to administration that better represent an institution's goals, and working with legal experts to utilize legal action when necessary. The best way for interest groups to influence the bureaucracy of the Executive Branch is through direct lobbying, which involves meeting with relevant bureaucrats and policymakers to make a case for the interest group`s policies and proposals. Interest groups can provide expert analysis and research to support their arguments and can also offer to collaborate with the bureaucracy on policy initiatives, and this allows them to participate in the rulemaking process. Using the bureaucracy of the Executive Branch, the American Dairy Association plans to monitor regulatory activity and identify rules or proposed rules that impact the introduction of flavored milk in schools. We would also contact and work alongside industry representatives, legal experts, and other stakeholders to submit comments on proposed regulations and advocate for changes that better align with the association`s goal of more milk for students.

\subsection{The Judicial Branch}
\hspace{\parindent} The Judicial Branch's decisions have a direct effect on public and private policy via their interpretations of the law. Generally, the best way for an interest group to influence the Judicial Branch is through public litigation and writing amicus curiae briefs to influence court decisions for issues that affect the group's interests. This affects how the judicial branch will rule on the issue, and it also affects the public's perception of the group and its policies. Per the Judicial Branch, the American Dairy Association will work with legal experts and dairy industry representatives to identify cases that impact the ability to introduce flavored milk in schools and have potential legal ramifications. Our policies should not conflict with the constitution, therefore the judicial branch will have no need to pass an unconstitutional judicial review. We will also write amicus curiae beliefs and other legal filings to influence court decisions that affect the introduction of flavored milk in schools.

\subsection{The Legislative Branch}
\hspace{\parindent} The Legislative Branch plays a significant role in policy-making with actions that include including and passing legislation, appropriating funds for government programs and initiatives, and conducting oversight over the executive branch. The best way for a group to influence the Legislative Branch is through direct lobbying, which involves meeting with relevant members of Congress and their staff to make a case for the group's policies and proposals, just like the bureaucracy of the Executive Branch. However, another common method to influence the Legislative Branch is through grassroots mobilization, which involves organizing rallies and protests, creating online petitions, and using social media to reach a broader audience. Through the Legislative Branch, the American Dairy Association will engage with members of Congress and their staff through regular meetings, hearings, and events to educate them about the benefits of flavored milk and its positive impact on children`s health as well as nutrition. We will also advocate for legislation that supports the introduction of flavored milk in schools, such as measures that provide funding for milk programs and remove restrictions on the types of milk that can be served. 

\section{The Conclusion}
\hspace{\parindent} As the American Dairy Association, we believe that pursuing our interests through the judicial branch would be the least effective way to achieve our goals. This is because the judicial branch is responsible for interpreting the law and deciding on cases brought before it. While it has the power to strike down laws that are deemed unconstitutional, this process is time-consuming and often unpredictable. In addition, the judicial branch may not be able to address all of the specific issues that the American Dairy Association is concerned with. It primarily deals with cases that have been brought before it and may not have the ability to initiate action or change policy directly. On the other hand, the executive branch, including the Presidency and the bureaucracy, has the power to shape policy and implement changes more directly. By working with administration officials and building relationships with influential members of government, we can advocate for our interests and push for changes that benefit the American Dairy Association as a collective. 

\hspace{\parindent} If we implemented this strategy in real life, we believe it would work for many reasons. Firstly, engaging with the executive branch is a common and effective approach used by interest groups to achieve their policy goals. The executive branch is responsible for implementing and enforcing laws and regulations, and has significant authority and resources to do so. By building relationships and working with influential members of the administration, interest groups can gain support and advocacy for their policy goals. Secondly, introducing flavored milk in schools is a policy that can be seen as a win-win for both dairy farmers and students. By promoting the health benefits of milk and introducing different flavors, interest groups can increase milk consumption among students while providing a new market for dairy farmers. This policy can be seen as a positive step towards promoting healthy lifestyles among young people, which would gain support from policymakers and the public.

\hspace{\parindent} Politics would play a significant role in the outcome of the American Dairy Association's goals. In order to be successful, it is important to understand the political landscape and to build relationships with key decision-makers in government. This includes members of Congress and their staff, as well as officials in the executive branch who are responsible for implementing and enforcing regulations. To be more successful in achieving our goals, we need to engage in grassroots efforts to raise awareness about the benefits of flavored milk in schools and build public support for our cause. We should also work to build coalitions with other groups that share our interests and goals, such as farmers' organizations and education groups. We can also use lobbying tactics to advocate for our goals and interests. This includes meeting with lawmakers and government officials to educate them about the benefits of flavored milk and why it is important for schools to offer it as an option. We can also use advertising and media outreach to raise public awareness about our goals and to pressure decision-makers to support our cause.
\pagebreak
\section{Oral Testimony in Congress}
\hspace{\parindent} Good morning, honorable members of the committee. I'm here today to represent the American Dairy Association, and we have an important proposal to introduce to you. As you all know, milk is a vital source of growth and development for our children. Unfortunately, students across the nation are stuck with the boring old plain milk flavor during school lunch. This deters them from drinking milk, thereby withholding their daily nutritional intake of a multitude of very important nutrients such as vitamin D, protein, calcium, and more. This means they're missing out on the health benefits that come with drinking milk. 

\hspace{\parindent} So, what's our solution to this problem? We propose introducing a new variety of flavors of milk in schools across the country. By giving students the option to choose from a variety of flavors, we believe we can significantly increase their intake of milk and, as a result, improve their overall health and nutrition.

\hspace{\parindent} Now, we understand that implementing this proposal will require the cooperation of all branches of government. We plan to work with the Executive Branch, including the Presidency and the bureaucracy, to identify and advocate for regulatory changes that will support the introduction of flavored milk in schools. We also plan to engage with members of Congress to advocate for legislation that supports our cause.

\hspace{\parindent} We believe that our proposal is not only good for the health and well-being of our children, but it also benefits the farmers who provide us with the milk. Introducing more flavored milk in schools will create new markets for dairy farmers, which will help to boost the agricultural industry and our local economies.

\hspace{\parindent} In conclusion, we urge you to consider our proposal and support our efforts to introduce a new variety of flavors of milk in schools across the country. Let's work together to provide our children with the nutrition they need and to support our farmers and our local communities. 

\hspace{\parindent} Thank you.

\end{document}