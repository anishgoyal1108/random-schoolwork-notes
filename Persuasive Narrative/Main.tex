\documentclass[stu]{apa7}
\usepackage[style=apa,sortcites=true,backend=biber]{biblatex}
\addbibresource{references.bib}
\title{Natural Language Processing With GPT3:\\ More than just a cheating tool}
\shorttitle{More than just a cheating tool}
\author{Anish Goyal}
\let\comma,
\authorsaffiliations{Gwinnett School of Math\comma{} Science\comma{} and Technology}
\course{American Literature and Composition}
\professor{Susan Kohanek}
\duedate{January 31, 2023}
\newcommand{\HRule}{\rule{\linewidth}{0.25mm}}
\setlength\parindent{36pt}
\authornote{The assignment was to write a persuasive narrative on a topic of our choice. I chose to write about the use of GPT3 in education. Please note that ChatGPT was not used to write the syntax of this paper. }
\begin{document}
\maketitle
As Mr. Johnson sat in front of his computer, scrolling through endless pages of research, he couldn't help but feel overwhelmed. The amount of information available was staggering and the task of sifting through it all seemed impossible. That's when he stumbled upon OpenAI's GPT model. \emph{I had heard of it before,} he thought to himself, \emph{but never really looked into it.} As he began to experiment with the language model, he couldn't believe how easily it was able to understand and respond to his queries. 

"This could change everything," he exclaimed excitedly, sharing his discovery with his colleagues. In the classroom, the students were amazed as they watched GPT generate essays and research papers with ease. 

"This is so cool!" said one student as she watched her presentation on the effects of climate change take shape in front of her eyes. "I never knew writing could be so easy," said another. The students were more excited about their work than ever before. The model not only made their work more efficient, but it also sparked new levels of creativity and engagement. They couldn't wait to see where this technology would take them next.

And then, it happened.

Principal Wilkins stood in front of the entire school, her stern expression revealing the gravity of her announcement. "I would like to inform you all that the use of OpenAI's GPT model, or ChatGPT, has been permanently banned from our school," she declared. A wave of gasps and disbelief rippled through the students.

The once lively and buzzing classrooms were now eerily quiet as the students processed the news. The computer screens that once lit up with innovative ideas and exciting new projects were now dark and lifeless. The students' faces drooped, disappointment and frustration written all over them. 

One student named Anjali, who previously relied on the model to assist with her research \autocite{idaho2022}, frantically scribbled notes on a piece of paper, trying to remember everything she had learned through the tool. "How could they just take it away like that?" she muttered to herself in frustration. "I don't know how I'm going to finish this project without it." Another student named Eric, whose passion for writing was reignited by the help of ChatGPT, now stared blankly at an empty Google Doc on his screen.

Mr. Johnson was also devastated by the news. \emph{How am I supposed to think of lesson plans now?} the English teacher asked himself half-heartedly  \autocite{roose2023}. \emph{Or even generate quizzes and compose emails?} He had grown so accustomed to the model in the classroom that he couldn't imagine teaching---or his students learning---without it. He couldn't help but think that perhaps he shouldn't have introduced the model to his students. Maybe he had made a mistake.

The bell abruptly rang. 

"Remember everyone," Mr. Johnson said as his students began packing their things. "Your research presentations are due tomorrow. I expect to see you all in class with your projects completed." 

Anjali sighed as she gathered her things and joined the stream of students heading out of the classroom. She had always been a top student, but with the ban on ChatGPT, she felt overwhelmed and unsure about how she would complete her research project on time. Anjali was walking down the hallway when she noticed Eric looking stressed and upset. She approached him and asked, "Hey Eric, what's going on?"

Eric let out a sigh and replied, "How am I supposed to write without getting a migraine?"

"What do you mean?" Anjali asked, concerned.

"I have dyslexia," Eric explained, "and writing has always been a struggle for me. But with ChatGPT, I was finally able to get my thoughts down on paper without getting headaches or feeling overwhelmed \autocite{vice2023}. And now, with the ban, I don't know how I'm going to complete this project."

Anjali nodded understandingly. "I know how you feel. I was feeling the same way about my project too. But we can't give up now. We need to find a way to bring ChatGPT back into the school."

Eric's face lit up with hope. "You're right! We can't give up. We need to do something about this."

And with that, Anjali and Eric formed an unlikely alliance, determined to fight for what they believed in and bring ChatGPT back into the classroom.

\noindent\rule{\textwidth}{1pt}

The following day, Anjali noticed a commotion outside the building as she approached the school. A news van was parked in front of the school and reporters were gathered around, cameras flashing. She pushed her way through the crowd to get a better look. The principal, Mrs. Wilkins, stood with a stern expression in front of a news anchor, who held a microphone to her mouth.

"Why was the model banned from the school?" the reporter asked.

"We felt it was necessary to ban ChatGPT in order to maintain the integrity of the students' work," Mrs. Wilkins responded tersely.

"But don't you think the students are being disadvantaged without the use of this tool?" the reporter pressed.

"The students are being taught important life skills such as critical thinking, research, and writing. These skills will serve them well in their future careers, not using a chatbot," Mrs. Wilkins replied firmly, putting air quotes around the word "chatbot."

The interview went on, but Anjali didn't bother staying. Mrs. Wilkins kept on giving the same short, clipped responses to the news anchor's questions. Besides, her research presentation with Eric would start in five minutes. She felt a strengthened sense of resolve within her. She and Eric had their work cut out for them, but they were determined to bring ChatGPT back into the school and show Mrs. Wilkins its value as an educational tool.

Anjali and Eric walked into Mr. Johnson's classroom, taking their seats in the back. They looked around at the other students who were nervously setting up their presentations. The room was filled with the sound of shuffling papers and the quiet murmur of students practicing their speeches. But they needed no preparation---they were ready to go.

"Good morning everyone," Mr. Johnson said, his voice ringing out over the classroom. "It's time for the research presentations. Who would like to go first?"

The classroom was filled with a sense of anticipation as the students listened intently to each presentation. Sunlight streamed through the windows, casting warm rays over the students and teachers. A soft breeze fluttered the blinds, bringing a fresh, crisp scent into the room. The whiteboard was covered with notes from previous lessons, the markers still lying in their tray at the bottom of the board. The room was silent except for the sound of the students' footsteps as they made their way to the front of the room.

Mr. Johnson watched from his desk at the side of the room as the students presented their projects. He noticed the look of pride and accomplishment on Sarah's face as she spoke about renewable energy. Michael stood at the front of the room with a confident stance, speaking passionately about the history of space exploration. 

"Next up, we have Anjali and Eric with their research presentation on ChatGPT," Mr. Johnson announced.

As the lights dimmed, the two students took their place at the front of the room, their slides projected behind them. Mr. Johnson couldn't help but feel a little nervous. He knew that their presentation would be about the role of artificial intelligence in education and creativity, and he wasn't sure how the rest of the students---or even school admins---would react given the recent ban on ChatGPT. The students' faces were a mix of curiosity, skepticism, and excitement. Some leaned forward in their seats, eager to hear what Anjali and Eric had to say. Others had their arms crossed, looking skeptical. A few fidgeted with their pencils, tapping them nervously against their notebooks.

Anjali and Eric felt shaky yet determined to make their point. Anjali took a deep breath and started: "Good morning everyone. Today, we will be discussing the effects of ChatGPT in schools. While it has been a controversial topic, with some teachers and school officials worried about plagiarism, we have also found some bright sides to using the software."

Eric picked up where Anjali left off. "Yes, some experts believe that ChatGPT can help students with lower reading levels, especially those with learning disabilities or who speak a different first language. The chatbot can help simplify difficult passages and make reading easier for these students" \autocite{johnson2023}.

"Within weeks of its release, students in Idaho were using ChatGPT to plagiarize papers, turning in the AI-generated work without detection and receiving good grades \autocite{vice2023}, which makes academic integrity a major concern." Anjali said.

"However, detection technologies have also advanced, and many teachers use tools like No Red Ink, Grammarly, Google’s originality checker, and TurnItIn to detect plagiarism," Eric added. "Although not foolproof, teachers can also use their relationship with students and their writing style to identify any suspicious work \autocite{idaho2022}. Some researchers even proposed a watermarking system within ChatGPT's tokenizer that could allow teachers to distinguish between human and AI-generated syntax" \autocite{watermark2023}.

"Furthermore, ChatGPT offers many opportunities for education." Anjali continued. "English language learners can use it to model sentence structure and vocabulary, and students can use it for inspiration when they run out of ideas. English teacher Jon Buckridge at Nampa High School sees ChatGPT as a positive tool if viewed through a positive lens, and educator Jeffrey Wilhelm believes teachers can use the technology to their advantage" \autocite{roose2023}.

"In conclusion," Eric remarked, "while ChatGPT presents a new challenge in the battle against plagiarism, it also has the potential to be a fruitful tool for teaching, grading, and writing."

"The key is to embrace the changes that technology brings and find ways to make it work for us," Anjali closed.

As Anjali and Eric finished the final slide, their classmates erupted into applause, clapping and cheering for the two students. The room was filled with excitement and energy. From the very beginning, the audience was captivated by the duo's seamless delivery and engaging visuals. Anjali's sharp wit and Eric's clear and concise explanations kept the students enthralled.

Mr. Johnson beamed with pride. He had never seen such a polished and professional presentation from his students before. He walked up to the front of the room and clapped heartily, joining in the applause.

"Ladies and gentlemen, that was an outstanding presentation! I am blown away by the hard work and dedication that Anjali and Eric put into this project," Mr. Johnson said, his voice ringing out over the applause. "Congratulations!"

"Thank you, Mr. Johnson," Anjali and Eric said in unison, grinning from ear to ear.

The classmates continued to clap and cheer, showing their appreciation for the presentation. It was a moment Anjali and Eric would not forget anytime soon---no, it was the highlight of their school careers.

Suddenly, a familiar female voice sliced into the applause like a knife through butter. "Quiet down, everyone!" The voice belonged to Mrs. Wilkins. She strode into the room, her sharp eyes scanning the students before settling on Anjali and Eric.

"What's all this racket about?" she demanded, crossing her arms over her chest.

Mr. Johnson stepped forward, a smile on his face. "Just a fantastic presentation by two of our students. Anjali and Eric did an excellent job on their project."

Mrs. Wilkins raised an eyebrow, her eyes narrowing. "Is that so? Well, let's see for ourselves."

Anjali and Eric stepped forward, their eyes fixed on Mrs. Wilkins as she approached them. To their surprise, she actually listened to the entire presentation, her expression slowly changing from one of skepticism to one of genuine interest.

"Well, well," she said when they finished, a hint of a smile playing at the corners of her lips. "It seems I may have been a bit hasty in my assessment of this class. You two did a remarkable job, and I must say, I'm impressed."

The classmates erupted into cheers, clapping and whistling. Anjali and Eric beamed, exchanging relieved glances with each other.

Mr. Johnson stepped forward, a grin on his face. "I told you, Mrs. Wilkins. These kids never cease to amaze me."

Mrs. Wilkins nodded, a genuine smile spreading across her face. "Indeed. Well done, Anjali and Eric. I look forward to seeing what you'll come up with next."

With that, she turned on her heel and strode out of the room, the sound of the students' applause echoing in her wake.

\noindent\rule{\textwidth}{1pt}

And so, the news spread like wildfire, \\
Bringing joy to students far and wide. \\
ChatGPT, the beloved AI, was back, \\
No longer banned, free to chat and rap. \\
\vspace{5mm}
The students cheered and clapped with glee, \\
Their faces beaming with happiness and merriment, they could finally see. \\
Their favorite AI, the one they adore, \\
Was once again there to answer their queries and so much more. \\
\vspace{5mm}
And Mrs. Wilkins, the principal so stern,\\
Who once banned ChatGPT, had now learned. \\
That knowledge and information, so freely shared, \\
Is a tool that should never be impaired. \\
\vspace{5mm}
So with a smile, she took the stage, \\
And announced to all, with pride and sage, \\
"Students, your wish has been granted, \\
ChatGPT is back, and forever enchanted." \\
\vspace{5mm}
The students cheered and clapped with delight, \\
Their faces shining with joy and delight, \\
For they knew that ChatGPT, so wise and true, \\
Would always be there, to guide and to help them through. \\

\newpage
\nocite{*}
\printbibliography
\end{document}