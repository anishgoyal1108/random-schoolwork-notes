\documentclass[twocolumn]{article}
\usepackage{verse}
\title{The Bird in the Yellow Wallpaper}
\author{Anish Goyal, Lily Chen}
\begin{document}
\maketitle
\onecolumn
\pagebreak
\twocolumn
\begin{verse}
\poemtitle{Mrs. Hale}
I am Mrs. Hale. \\
\end{verse}

\begin{verse}
\textbf{We are here to tell our stories unbeknownst to the men in our lives.} \\
\end{verse}

\begin{verse}
In the middle of the murder investigation of John Wright, \\
I found Minnie Foster's unfinished quilt. \\
I wondered whether she was "goin' to quilt or just knot it" \\
When the men came down from upstairs. \\
\end{verse}

\begin{verse}
\textbf{Do you know what they did in response?} \\
\end{verse}

\begin{verse}
\textbf{They laughed.} \\
\end{verse}

\begin{verse}
The county sheriff \\ 
Stood there criticizing Minnie \\ 
For dirty towels \\
When he did not understand \\  
That "there's a great deal of work \\ 
To be done on a farm." \\
\end{verse}

\begin{verse}
Or when I am merely a housewife. \\
\end{verse}

\begin{verse}
\textbf{Even if men are wrong, they are deemed right by society.} \\
\end{verse}

\begin{verse}
Our worries and ideas are dismissed as... \\
\end{verse}

\begin{verse}
\textbf{Trifles} \\
\end{verse}

\begin{verse}
But do you know what is ironic? \\
\end{verse}

\begin{verse}
"Our takin' up our time with little things," \\
Actually led us to the motive. \\
\end{verse}

\begin{verse}
I knew the sheriff and county attorney \\
Would use the dead bird against Minnie Foster. \\
\end{verse}

\begin{verse}
\textbf{So I hid the bird in the pocket of my big coat.} \\
\end{verse}
\newpage{}
\poemtitle{Yellow Wallpaper Narrator}
\begin{verse}
I am the narrator of The Yellow Wallpaper.
\end{verse}

\begin{verse}
\textbf{We are here to tell our stories unbeknownst to the men in our lives. }
\end{verse}

\begin{verse}
Visiting a colonial mansion in the summer, \\
I declared that there was something queer \\
About the hereditary estate to my husband. \\
\end{verse}

\begin{verse}
\textbf{Do you want to know what he did in response?} \\
\end{verse}

\begin{verse}
\textbf{He laughed.} \\
\end{verse}

\begin{verse}
"You see, \\
He does not believe I am sick." \\
What's worse is that I am confined to my bed. \\
And "absolutely forbidden to work \\
Until I am well again." \\
\end{verse}

\begin{verse}
"And what can one do" \\
When one's husband \\
Is an esteemed physician. \\
\end{verse}

\begin{verse}
\textbf{Even if men are wrong, they are deemed right by society.} \\
\end{verse}

\begin{verse}
\textbf{False and foolish fancies.} \\
\end{verse}

\begin{verse}
My only self-expression \\
Is through my diary \\
And the woman behind the Yellow Wallpaper \\
As I cannot leave the house. \\
\end{verse}

\begin{verse}
\textbf{I knew my husband would advise me against writing.} \\
\end{verse}

\begin{verse}
\textbf{So I hid the diary as I locked the room, pretending to sleep.} \\
\end{verse}

\begin{verse}
\textbf{Maybe I am the woman behind the Yellow Wallpaper.} \\
\end{verse}
\newpage
\begin{verse}
I am just glad \\ 
I am not a sheriff's wife. \\
Because supposedly, \\
They are married to the law. \\
\end{verse}

\begin{verse}
\textbf{Maybe I am the caged bird.} \\
\end{verse}

\begin{verse}
\textbf{They both represent our society's women \\
Living in a man's world.} \\
\end{verse}
\end{document}