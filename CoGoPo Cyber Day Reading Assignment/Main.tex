\documentclass[stu]{apa7}
\usepackage{multicol}
\title{Cyber Day \#1 Reading Assignment}
\author{Anish Goyal} 
\let\comma,
\authorsaffiliations{Gwinnett School of Math\comma{} Science\comma{} and Technology}
\course{AP Comparative Government and Politics}
\professor{William Cossen}
\duedate{March 1, 2023}
\newcommand{\HRule}{\rule{\linewidth}{0.25mm}}
\setlength\parindent{36pt}
\authornote{The assignment was to use the reading material on the significant changes to Mexico`s National Electoral Institute to create a short reflection on the article in the form of a K/W/L exercise.}
\newenvironment{threecolumns}{
  \begin{multicols}{3}
  \setlength{\columnsep}{1.5cm}
}{
  \end{multicols}
}
\begin{document}
\maketitle

\begin{threecolumns}
    \section{Know}
    \begin{itemize}
    \item Mexican lawmakers passed measures overhauling the electoral agency. 
    \item The changes include the reduction of the electoral agency`s staff and an attempt to make voting more efficient for Mexicans who live abroad.
    \item The leader of the President`s party in the Senate has called the measures unconstitutional.
    \item The National Electoral Institute earned international acclaim for facilitating clean elections in Mexico.
    \end{itemize}
        
    \section{Want to Know}
    \begin{itemize}
    \item What is the current state of democracy in Mexico?
    \item What will be the effect of the measures passed by the Mexican lawmakers on the forthcoming election in the country?
    \item What is the reaction of the Supreme Court to these measures?
    \item What are the concerns of the critics of the Mexican president`s administration? 
    \item What is the history of the electoral agency in Mexico?
    \end{itemize}

    \section{Learned}
    \begin{itemize}
    \item The Mexican president is undermining democratic norms.
    \item The changes made by the lawmakers will limit the electoral agency`s autonomy and its ability to punish politicians.
    \item Critics believe that this is an attempt to weaken a key pillar of Mexican democracy.
    \item The measures adopted will force electoral officials to eliminate thousands of jobs.
    \item This could result in the failure to install a significant number of polling stations, depriving thousands or hundreds of thousands of \item people of the right to vote.
    \item The Supreme Court is expected to hear a challenge to the measures in the coming months.
    \item The changes limit the agency`s control over its own spending and prevent it from disqualifying candidates for campaign spending violations. 
    \item The National Electoral Institute helped push the country away from one-party rule two decades ago.
    \item The president`s resentment of the electoral authority makes him act irrationally on this issue.
    \end{itemize}
\end{document}