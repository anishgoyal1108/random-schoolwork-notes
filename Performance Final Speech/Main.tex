\documentclass[stu, donotrepeattitle]{apa7}
\usepackage[style=apa,sortcites=true,backend=biber]{biblatex}
\title{Reflections on Courage and Cowardice: Exploring the Depths of Bravery in \textit{The Things They Carried}}
\shorttitle{Performance Final Speech}
\author{Anish Goyal}
\let\comma,
\authorsaffiliations{Gwinnett School of Math\comma{} Science\comma{} and Technology}
\course{American Literature and Composition}
\professor{Susan Kohanek}
\duedate{May 24, 2023}
\newcommand{\HRule}{\rule{\linewidth}{0.25mm}}
\setlength\parindent{36pt}
\authornote{The assignment was to select an aphorism, quotation, or excerpt from an American literary work, and explain why the selected material appealed to me and what it meant to me.}
\begin{document}
\maketitle
\textbf{``I was a coward. I served in the military.``}\\~\\
This quote comes from one of the most moving passages in the last chapter of Tim O' Brien's postmodern psychological fiction novel \textit{The Things They Carried}. The metafictional Tim writes this while contemplating his experiences in the Vietnam War, reflecting on the after effects of war and attempting to come to terms with his own cowardice. The construction of this line is seemingly straightforward, and yet the meaning it conveys is deep. It goes to the core of the myriad of conflicting feelings that entail serving one's nation in a time of war and highlights the sacrifices that troops make in order to protect their country. \\~\\
\textbf{``I was a coward. I served in the military.``}\\~\\
Furthermore, this remark serves as a good reminder to us that showing bravery does not always mean engaging in activities that require a high level of physical fortitude. It takes just as much courage to face our own fears and doubts, to make an effort to understand our own motives, and to wrestle with our own constraints in an honest way. It further emphasizes that even if we are unable to meet these difficulties, we can still find strength in the fact that we attempted to do so.\\~\\
\textbf{``I was a coward. I served in the military.``}\\~\\
Finally, this quote alludes to a larger aspect of the human experience that goes beyond the realm of war and conflict. It reminds us that we are all capable of becoming cowards, of coming up with excuses, and of avoiding the obligations that we have. However, it also shows that bravery is not an absence of dread, but rather the capacity to confront that fear head-on, to realize our own weaknesses, and to act despite these disparities. The takeaway from O'Brien's comment is quite clear: the concept of courage isn't only about what we do; it's also about who we are. This is true regardless of whether we're confronting our darkest fears or just attempting to act responsibly in our day-to-day lives.\\~\\
\textbf{``I was a coward. I served in the military.``}

\end{document}